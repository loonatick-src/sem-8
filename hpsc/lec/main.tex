\documentclass[a4paper]{article}

\usepackage[utf8]{inputenc}
\usepackage[T1]{fontenc}
\usepackage{textcomp}
\usepackage[english]{babel}
\usepackage{amsmath, amssymb}
\usepackage{physics}


% figure support
\usepackage{import}
\usepackage{xifthen}
\pdfminorversion=7
\usepackage{pdfpages}
\usepackage{transparent}
\newcommand{\incfig}[1]{%
	\def\svgwidth{\columnwidth}
	\import{./figures/}{#1.pdf_tex}
}

\pdfsuppresswarningpagegroup=1
\title{ME 766 - High Performance Scientific Computing}
\begin{document}
\section{Introduction}
Course content
\begin{itemize}
	\item Introduction to HPSC and scientific computing
	\item Processor performance, memory hierarchy, multi-core computing and vector computing
	\item Introduction  to parallel programming concepts and parallel algorithms
	\item Effective use of Bash scripting
	\item Effective use of tools like git, SVN, Mercurial etc
	\item OpenMP, MPI, GPGPU, Vector programming
	\item debuggers
	\item Performance analysis
	\item Numerical methods, applications of 
\end{itemize}

This course is all about making you a better researcher. It
is very hands on. This course
gets me rock hard. The processors keep changing every two years.
How to exploit hardware-level optimization techniques will also
change every two years LOL. This course is targeted at those who are
pursuing a project - help them improve their code over there. So,
include this in your SoPs and course description. 

The biggest component of this course the project. All the best.
This project is a distillation of the professor's experience over
B.Tech. This man wants to \emph{teach}. So you better \emph{imbibe}. 


\end{document}
